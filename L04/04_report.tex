\documentclass[11pt,a4paper, uplatex]{jsarticle}
%
\usepackage{amsmath,amssymb}
\usepackage{bm}
\usepackage[dvipdfmx]{graphicx}
\usepackage{ascmac}
\usepackage{listings}
\usepackage{underscore}
\lstset{
    frame=single,
    numbers=left,
    tabsize=2
}
%
\setlength{\textwidth}{\fullwidth}
\setlength{\textheight}{40\baselineskip}
\addtolength{\textheight}{\topskip}
\setlength{\voffset}{-0.2in}
\setlength{\topmargin}{0pt}
\setlength{\headheight}{0pt}
\setlength{\headsep}{0pt}
%
\newcommand{\divergence}{\mathrm{div}\,}  %ダイバージェンス
\newcommand{\grad}{\mathrm{grad}\,}  %グラディエント
\newcommand{\rot}{\mathrm{rot}\,}  %ローテーション
%
\title{プログラミング言語実験・C言語 第4回課題レポート}
\author{1510151  栁 裕太}
\date{\today}
\begin{document}
\maketitle
\section{課題7}
%
\subsection{スクリーンショット}

\begin{figure}[h]
  \centering
  \caption{実際にペアが出された時のスクリーンショット}
  \includegraphics[width=120mm]{ex07.eps}
\end{figure}

\subsection{配列に対する操作}
\subsubsection{make_info_table()}
第一引数info_cards配列を初期化したのち、各数字におけるカードの枚数(第二引数my_cards配列1-3行目各列の合計)を、info_table配列4行目に記録する。
\subsubsection{search_low_pair()}
まず提出するカードを示す第一引数dst_cards配列を初期化する。

その後、第二引数info_table配列4行目をfor文で要素が2以上(=ペアが存在する)の部分が存在するまでループを回し、
もしペアが存在するならば、該当する列の第三引数my_cards配列の各行要素をdst_cards配列に代入し、1を返すことで処理を終了する。

もしペアが存在しないならば、0を返すことで処理を終了する。
\subsubsection{select_cards_free()}
まず、info_table配列を定義した後、make_info_table関数を呼び出し、info_table配列と第二引数my_cards配列を渡すことで、ペアの情報を取得する。

その後、count_cards()関数を呼び出し、第一引数select_cards配列を渡すことで、出すカードが決定されている状態であるかを調べる。
もし出すカードが決まっていないのならば、search_low_pair()関数を呼び出し、出せられるペアの中で最弱のものを提出カードに指定する。
ここでsearch_low_pair()関数にて出せられるペアが存在しないのならば、search_low_cards()関数を呼び出し単騎カードを提出カードに指定する。

\subsection{機能実装の決め手}
\subsubsection{make_info_table()}
for文により、my_cards配列0~3行目の任意の列の合計をinfo_table配列4行目の同じ列番号に代入することで、ペア情報の作成が可能となっている。
\subsubsection{search_low_pair()}
まず、for文でinfo_table配列4行目にてペアを出せられる数字があるかを1行目から調べ、i列に存在した時点でbreakしている。
これにより、その直後にif(i<=13)とすることで、ペアが存在しない = iが14 (= ペアが存在すれば14に達する前にbreakされている)ケースは、
ペアを指定するシーケンスをパスすることが可能となっている。

\subsubsection{select_cards_free()}
提出カードを指定するときに、search_low_pair()関数或いはsearch_low_card()関数を呼び出す前に、
if(count_cards(select_cards)==0)を活用している。これにより、先に提出カードにペアが指定された後に、
単騎提出カードに上書き指定されるのを防ぐことが可能となっている。
\section{課題8}
\subsection{スクリーンショット}
\begin{figure}[h]
  \centering
  \caption{実際に階段が出された時のスクリーンショット}
  \includegraphics[width=120mm]{ex08.eps}
\end{figure}
\subsection{配列に対する操作}
\subsection{機能実装の決め手}
%
%
\end{document}
