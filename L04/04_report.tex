\documentclass[11pt,a4paper, uplatex]{jsarticle}
%
\usepackage{amsmath,amssymb}
\usepackage{bm}
\usepackage[dvipdfmx]{graphicx}
\usepackage{ascmac}
\usepackage{listings}
\usepackage{underscore}
\lstset{
    frame=single,
    numbers=left,
    tabsize=2
}
%
\setlength{\textwidth}{\fullwidth}
\setlength{\textheight}{40\baselineskip}
\addtolength{\textheight}{\topskip}
\setlength{\voffset}{-0.2in}
\setlength{\topmargin}{0pt}
\setlength{\headheight}{0pt}
\setlength{\headsep}{0pt}
%
\newcommand{\divergence}{\mathrm{div}\,}  %ダイバージェンス
\newcommand{\grad}{\mathrm{grad}\,}  %グラディエント
\newcommand{\rot}{\mathrm{rot}\,}  %ローテーション
%
\title{プログラミング言語実験・C言語 第4回課題レポート}
\author{1510151  栁 裕太}
\date{\today}
\begin{document}
\maketitle
\section{課題7}
%
\subsection{スクリーンショット}

\begin{figure}[h]
  \centering
  \caption{実際にペアが出された時のスクリーンショット}
  \includegraphics[width=120mm]{ex7-scs.png}
\end{figure}

\subsection{配列に対する操作}

\subsection{機能実装の決め手}

\section{課題8}
\subsection{スクリーンショット}
\begin{figure}[h]
  \centering
  \caption{実際に階段が出された時のスクリーンショット}
  \includegraphics[width=120mm]{ex8-scs.png}
\end{figure}
\subsection{配列に対する操作}
\subsection{機能実装の決め手}
%
%
\end{document}
