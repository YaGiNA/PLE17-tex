\documentclass[11pt,a4paper, uplatex]{jsarticle}
%
\usepackage{amsmath,amssymb}
\usepackage{bm}
\usepackage{graphicx}
\usepackage{ascmac}
\usepackage{listings}
\lstset{
    frame=single,
    numbers=left,
    tabsize=2
}
%
\setlength{\textwidth}{\fullwidth}
\setlength{\textheight}{40\baselineskip}
\addtolength{\textheight}{\topskip}
\setlength{\voffset}{-0.2in}
\setlength{\topmargin}{0pt}
\setlength{\headheight}{0pt}
\setlength{\headsep}{0pt}
%
\newcommand{\divergence}{\mathrm{div}\,}  %ダイバージェンス
\newcommand{\grad}{\mathrm{grad}\,}  %グラディエント
\newcommand{\rot}{\mathrm{rot}\,}  %ローテーション
%
\title{プログラミング言語実験・C言語 第3回課題レポート}
\author{1510151  栁 裕太}
\date{\today}
\begin{document}
\section{課題5}
%
\subsection{サーバーのみ実行時}
%
\subsection{クライアント実行時}
%
\section{課題6}
%
\subsection{common.c}
%
\subsubsection{copy\_table()}
\paragraph{何の処理?}
\paragraph{配列への処理内容}
\paragraph{機能実現への決め手}

%
\subsubsection{clear\_table()}
\paragraph{何の処理?}
\paragraph{配列への処理内容}
\paragraph{機能実現への決め手}
%
\subsubsection{copy\_cards()}
\paragraph{何の処理?}
\paragraph{配列への処理内容}
\paragraph{機能実現への決め手}
%
\subsubsection{clear\_cards()}
\paragraph{何の処理?}
\paragraph{配列への処理内容}
\paragraph{機能実現への決め手}
%
\subsubsection{diff\_cards()}
\paragraph{何の処理?}
\paragraph{配列への処理内容}
\paragraph{機能実現への決め手}
%
\subsubsection{or\_cards()}
\paragraph{何の処理?}
\paragraph{配列への処理内容}
\paragraph{機能実現への決め手}
%
\subsubsection{and\_cards()}
\paragraph{何の処理?}
\paragraph{配列への処理内容}
\paragraph{機能実現への決め手}
%
\subsubsection{not\_cards()}
\paragraph{何の処理?}
\paragraph{配列への処理内容}
\paragraph{機能実現への決め手}
%
\subsubsection{get\_field\_state\_from\_field\_cards()}
\paragraph{何の処理?}
\paragraph{配列への処理内容}
\paragraph{機能実現への決め手}
%
\subsubsection{get\_field\_state\_from\_own\_cards()}
\paragraph{何の処理?}
\paragraph{配列への処理内容}
\paragraph{機能実現への決め手}
%
\subsubsection{remove\_low\_card()}
\paragraph{何の処理?}
\paragraph{配列への処理内容}
\paragraph{機能実現への決め手}
%
\subsubsection{remove\_suit()}
\paragraph{何の処理?}
\paragraph{配列への処理内容}
\paragraph{機能実現への決め手}
%
\subsubsection{count\_cards()}
\paragraph{何の処理?}
\paragraph{配列への処理内容}
\paragraph{機能実現への決め手}
%
\subsection{select\_cards.c}
%
\subsubsection{select\_change\_cards()}
\paragraph{何の処理?}
\paragraph{配列への処理内容}
\paragraph{機能実現への決め手}
%
\subsubsection{select\_submit\_cards()}
\paragraph{何の処理?}
\paragraph{配列への処理内容}
\paragraph{機能実現への決め手}
%
\subsubsection{select\_cards\_restrict()}
\paragraph{何の処理?}
\paragraph{配列への処理内容}
\paragraph{機能実現への決め手}
%
\subsubsection{select\_cards\_free()}
\paragraph{何の処理?}
\paragraph{配列への処理内容}
\paragraph{機能実現への決め手}

\end{document}
