\documentclass[11pt,a4paper, uplatex]{jsarticle}
%
\usepackage{amsmath,amssymb}
\usepackage{bm}
\usepackage{graphicx}
\usepackage{ascmac}
\usepackage{listings}
\lstset{
    frame=single,
    numbers=left,
    tabsize=2
}
%
\setlength{\textwidth}{\fullwidth}
\setlength{\textheight}{40\baselineskip}
\addtolength{\textheight}{\topskip}
\setlength{\voffset}{-0.2in}
\setlength{\topmargin}{0pt}
\setlength{\headheight}{0pt}
\setlength{\headsep}{0pt}
%
\newcommand{\divergence}{\mathrm{div}\,}  %ダイバージェンス
\newcommand{\grad}{\mathrm{grad}\,}  %グラディエント
\newcommand{\rot}{\mathrm{rot}\,}  %ローテーション
%
\title{プログラミング言語実験・C言語 第2回課題レポート}
\author{1510151  栁 裕太}
\date{\today}
\begin{document}
\maketitle
%
%
\section{課題3}
\subsection{ソースコード}
\begin{lstlisting}[language=c]
#include <stdio.h>
#include <stdlib.h>
#include <math.h>
#include <ctype.h>

#define TRUE 0
#define FALSE 1

typedef char data_type; /* データの型 */
typedef struct node_tag {
    data_type data; /* データ */
    struct node_tag *next; /* 後続ノードへのポインタ */
} node_type; /* ノードの型 */

void error(char *msg) {
    fprintf(stderr, "error: %s\n", msg);
    exit(1);
}

void initialize(node_type **pp) /* スタックの初期化 */
{
    *pp = NULL; /* スタックは空(先頭ノードなし) */
}

node_type *new_node(data_type x, node_type *p)
{
    node_type *temp;
    temp = (node_type *)malloc(sizeof(node_type));
    /* メモリの割当て */
    if (temp == NULL) return NULL; /* メモリ割当て失敗 */
    else { /* ノードの各メンバーの値の設定 */
    temp->data = x;
    temp->next = p;
    return temp;
    }
}

int is_empty(node_type *p) /* 空スタックのとき真、 */
{ /* そうでないならば偽を返す */
    if (p == NULL) return TRUE; /* 空スタックのとき */
    else return FALSE; /* 空スタックでないとき */
}

data_type top(node_type *p)
{
    if (p == NULL) /* 空スタックのとき */
      return '\0'; /* ナル文字を返す */
    else /* 空スタックでないとき */
      return p->data; /* スタックの先頭のデータを返す */
}

int push(node_type **pp, data_type x)
{
    node_type *temp;
    temp = new_node(x, *pp); /* 関数new_nodeの呼出し */
    if (temp == NULL) return 1;
    *pp = temp;
    return 0;
}

int pop(node_type **pp) {
    node_type *temp;
    if (*pp != NULL) {
        temp = (*pp)->next;
        free(*pp); /* メモリの解放 */
        *pp = temp;
        return 0;
    }
    else return 1;
}


// 引用元: http://home.a00.itscom.net/hatada/c-tips/rpn/rpn02.html
// 演算子の優先順位を返す
int rank(char *op) {
    if (*op == '*' || *op == '/' ) return 3;
    if (*op == '+' || *op == '-') return 2;
    if (*op == '(') return 4;
    if (*op == ')') return 1;
    if (*op == '=') return 0;
    return 5;
}


void convert2(char *token[], int length) {
    node_type *stack;
    initialize(&stack);
    int n;
    data_type head_stack[1];
    for (n = 0; n < length; n++) {
        head_stack[0] = top(stack);
        while(is_empty(stack) != TRUE &&
              '(' != head_stack[0] &&
              rank(token[n]) <= rank(head_stack)){
            printf("%c ", top(stack));
            pop(&stack);
            head_stack[0] = top(stack);
        }
        if(*token[n] != ')'){
            push(&stack, *token[n]);
        }else{
            pop(&stack);
        }
    }
    while(is_empty(stack) != TRUE){
        printf("%c ", top(stack));
        pop(&stack);
    }
    printf("\n");
}


// 引用終了

int main(void){
    printf("B=2, C=3, D=4, E=5, F=6として演算\n");
    printf("-------------------------\n");
    data_type *frml1[] = {
        "A", "=", "(", "2", "-", "3", ")",
        "/", "4", "+", "5", "*", "6"
    };
    convert2(frml1, sizeof(frml1)/sizeof(char*));
    printf("A=%f\n", (2.0-3.0)/4.0+5.0*6.0);
    printf("-------------------------\n");
    data_type *frml2[] = {
        "A", "=", "(", "2", "-", "3", "/",
        "4", "+", "5", ")", "*", "6"
    };
    convert2(frml2, sizeof(frml2)/sizeof(char*));
    printf("A=%f\n", (2.0-3.0/4.0+5.0)*6.0);
    printf("-------------------------\n");
    data_type *frml3[] = {
        "A", "=", "2", "-", "3", "/", "(",
        "4", "+", "5", "*", "6" ")",
    };
    convert2(frml3, sizeof(frml3)/sizeof(char*));
    printf("A=%f\n", 2.0-3.0/(4.0+5.0*6.0));
}
\end{lstlisting}
\subsection{実行結果}
\begin{lstlisting}
% ./ex3_rev_pol_not.c
  B=2, C=3, D=4, E=5, F=6として演算
  -------------------------
  A 2 3 - 4 / 5 6 * + =
  A=29.750000
  -------------------------
  A 2 3 4 / - 5 + 6 * =
  A=37.500000
  -------------------------
  A 2 3 4 5 6 * + ( / - =
  A=1.911765
\end{lstlisting}
\section{課題4}
\subsection{ソースコード}
\begin{lstlisting}[language=c]
#include <stdio.h>
#include <stdlib.h>
#include <math.h>

#define SUCCESS 0
#define FAILURE 0

typedef double data_type;
typedef struct node_tag {
    data_type data;
    struct node_tag *left;
    struct node_tag *right;
} node_type; /* ノードの型 */


void initialize(node_type **pp){
    *pp = NULL;
}


node_type *new_node(data_type x) {
    node_type *temp;
    temp = (node_type *)malloc(sizeof(node_type));
    if (temp == NULL) return NULL;
    else {
        temp->data = x;
        temp->left = NULL;
        temp->right = NULL;
        return temp;
    }
}


int insert(node_type **pp, data_type x) {
    node_type *temp;
    if (*pp == NULL) {
        temp = new_node(x);
        if (temp == NULL) return FAILURE;
        *pp = temp;
        return SUCCESS;
    }else {
        /* 絶対値比較のためfabs()使用 */
        if (fabs(x) <= fabs((*pp)->data))
            insert(&((*pp)->left), x);
        else if (fabs(x) > fabs((*pp)->data))
            insert(&((*pp)->right), x);
    }
    return SUCCESS;
}


double tree_size(node_type *p) {
    double sump=0, left=0, right=0;
    /* 子ノードが存在すれば、再帰呼び出しで子ノードまでの合計を呼び出す */
    if(p->left != NULL){
        left=tree_size(p->left);
    }
    printf("%g\n", p->data);
    if(p->right != NULL){
        right=tree_size(p->right);
    }
    /* 合計は、左右の子ノードまでの合計+自分の値 */
    sump = left + right + p->data;
    return sump;
}



int main(void){
    node_type *root;
    initialize(&root);
    double num, sum_sorted = 0;
    while(1){
        scanf("%lf", &num);
        if(num == 0){break;}
        insert(&root, num);
    }

    printf("----------------sorted by fabs()----------------\n");
    sum_sorted = tree_size(root);
    printf("\n");
    printf("sorted sum: %g\n", sum_sorted);

    // list_free(root);
    return 0;
}
\end{lstlisting}
\subsection{実行結果}
\begin{lstlisting}
% ./ex4_bin_ser_tree.out < ./numbers.txt
  ----------------sorted by fabs()----------------
  -0.002
  -0.01
  0.012
  0.3
  3.1
  -6.4
  7
  8
  10
  23
  -30
  46
  -70
  -100
  360
  -500
  -1700
  -3000
  5000
  -1e+16
  1e+16
  sorted sum: 52

\end{lstlisting}
%
%
\end{document}
