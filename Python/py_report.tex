\documentclass[11pt,a4paper, uplatex]{jsarticle}
%
\usepackage{amsmath,amssymb}
\usepackage{bm}
\usepackage[dvipdfmx]{graphicx}
\usepackage{ascmac}
\usepackage{listings}
\usepackage{underscore}
\lstset{
    frame=single,
    numbers=left,
    tabsize=2
}
%
\setlength{\textwidth}{\fullwidth}
\setlength{\textheight}{40\baselineskip}
\addtolength{\textheight}{\topskip}
\setlength{\voffset}{-0.2in}
\setlength{\topmargin}{0pt}
\setlength{\headheight}{0pt}
\setlength{\headsep}{0pt}
%
\newcommand{\divergence}{\mathrm{div}\,}  %ダイバージェンス
\newcommand{\grad}{\mathrm{grad}\,}  %グラディエント
\newcommand{\rot}{\mathrm{rot}\,}  %ローテーション
%
\title{プログラミング言語実験・Python 課題レポート}
\author{1510151  栁 裕太}
\date{\today}
\begin{document}
\maketitle
\section{課題1}
\subsection{課題1-1 実験結果}
実行結果テキスト: https://goo.gl/hsNwv9
%
\subsection{課題1-2 実験結果}
実行結果テキスト: https://goo.gl/Ub99j3
%
\section{課題2}
\subsection{課題2-1 実験結果}
実行結果音声: https://goo.gl/nHjB7b
%
\subsection{課題2-2 実験結果}
実行結果画像: https://goo.gl/HcXeHH
%
\section{課題3}
\subsection{課題3-1 実験結果}
実行結果テキスト: https://goo.gl/tCwXMp
%
\subsection{課題3-2 実験結果}
実行結果音声: https://goo.gl/krFJQF

\section{課題4}
\subsection{課題4 実験結果}
\subsubsection{処理前音声プロット結果}
HAL0001noiseColored.wav: https://goo.gl/9eiFYw

HAL0002noiseColored.wav: https://goo.gl/B74rsM

KnoiseColored.wav: https://goo.gl/UvwktX
%
\subsubsection{周波数を切った場所}
答え: 4KHz

理由: 3種のパワースペクトルのグラフにおける、10KHzを山とするホワイトノイズが消える
(中央左右におけるVoiceと思われる部分と交わる)場所が4KHzの場所だったため
%
\subsubsection{処理後音声ファイル}
HAL0001noiseCleared.wav: https://goo.gl/kb5P7v

HAL0002noiseCleared.wav: https://goo.gl/URiKdK

KnoiseCleared.wav: https://goo.gl/pJ6Gcj
%
\subsubsection{処理後音声プロット結果}
HAL0001noiseCleared.wav: https://goo.gl/uwPRnR

HAL0002noiseCleared.wav: https://goo.gl/WNqKnb

KnoiseCleared.wav: https://goo.gl/ag8GnT
%

\subsubsection{何と言っているか}
what_are_they_saying.md: https://goo.gl/GRtTaw
\end{document}
